\documentclass{article}
\usepackage[utf8]{inputenc}

\title{PS5}
\author{Audrey Hopewell }
\date{2/20/2020}

\begin{document}

\maketitle

\section{Question 3}
I gathered data from the Wikipedia page of Richard Linklater, a movie director I like. Specifically, I wanted the data set of his feature films and the roles he served for each one since he often writes, directs, and produces a film. For my master's research, I'm looking at social networks in the movie industry, so I'm hoping this method of gathering data will be helpful. This was basically what we did in class, so I didn't use any online tutorials.

\section{Question 4}
I gathered data from Twitter by searching for tweets within 200 miles of Norman that used the hashtag "Oscars2020" using the package twitteR. There were 1,176 tweets in this range. Then I converted to a data frame that contained 16 variables, including the text of the tweet, username, whether it was a retweet, etc. I was just curious to see how many of the tweets were retweets (314) rather than original tweets, and how many mentioned the Best Picture winner, Parasite (213). The maximum number of favorites was 232 and the maximum number of retweets was 69, so no one from the area had a viral tweet about the Oscars.



\end{document}
