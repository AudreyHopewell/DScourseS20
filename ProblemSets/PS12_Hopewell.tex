\documentclass{article}
\usepackage[utf8]{inputenc}

\title{PS 12}
\author{Audrey Hopewell }
\date{April 30, 2020}

\usepackage{natbib}
\usepackage{graphicx}

\begin{document}

\maketitle

\section{Question 6}
\begin{table}[!htbp] \centering 
  \caption{} 
  \label{} 
\begin{tabular}{@{\extracolsep{2pt}}lccccccc} 
\\[-1.8ex]\hline 
\hline \\[-1.8ex] 
Statistic & \multicolumn{1}{c}{N} & \multicolumn{1}{c}{Mean} & \multicolumn{1}{c}{St. Dev.} & \multicolumn{1}{c}{Min} & \multicolumn{1}{c}{Pctl(25)} & \multicolumn{1}{c}{Pctl(75)} & \multicolumn{1}{c}{Max} \\ 
\hline \\[-1.8ex] 
logwage & 1,545 & 1.652 & 0.688 & $-$0.956 & 1.201 & 2.120 & 4.166 \\ 
hgc & 2,229 & 12.455 & 2.444 & 5 & 11 & 14 & 18 \\ 
exper & 2,229 & 6.435 & 4.867 & 0.000 & 2.452 & 9.778 & 25.000 \\ 
kids & 2,229 & 0.429 & 0.495 & 0 & 0 & 1 & 1 \\ 
\hline \\[-1.8ex] 
\end{tabular} 
\end{table} 

over 30\% of logwages are missing. They are most likely missing not at random because women with lower wages (related to fewer years of school or experience) are probably less likely to report their earnings.

\newpage
\section{Question 7}
\begin{table}[!htbp] \centering 
  \caption{Results} 
  \label{} 
\begin{tabular}{@{\extracolsep{5pt}}lccc} 
\\[-1.8ex]\hline 
\hline \\[-1.8ex] 
 & \multicolumn{3}{c}{\textit{Dependent variable:}} \\ 
\cline{2-4} 
\\[-1.8ex] & \multicolumn{3}{c}{logwage} \\ 
\\[-1.8ex] & \multicolumn{2}{c}{\textit{OLS}} & \textit{selection} \\ 
\\[-1.8ex] & (1) & (2) & (3)\\ 
\hline \\[-1.8ex] 
 hgc & 0.059$^{***}$ & 0.036$^{***}$ & 0.091$^{***}$ \\ 
  & (0.009) & (0.006) & (0.010) \\ 
  & & & \\ 
 union1 & 0.222$^{**}$ & 0.068 & 0.186$^{**}$ \\ 
  & (0.087) & (0.047) & (0.084) \\ 
  & & & \\ 
 college1 & $-$0.065 & $-$0.126$^{***}$ & 0.092 \\ 
  & (0.106) & (0.048) & (0.100) \\ 
  & & & \\ 
 exper & 0.050$^{***}$ & 0.021$^{***}$ & 0.054$^{***}$ \\ 
  & (0.013) & (0.007) & (0.012) \\ 
  & & & \\ 
 I(exper$\hat{\mkern6mu}$2) & $-$0.004$^{***}$ & $-$0.001$^{***}$ & $-$0.002$^{*}$ \\ 
  & (0.001) & (0.0004) & (0.001) \\ 
  & & & \\ 
 Constant & 0.834$^{***}$ & 1.149$^{***}$ & 0.446$^{***}$ \\ 
  & (0.113) & (0.078) & (0.122) \\ 
  & & & \\ 
\hline \\[-1.8ex] 
Observations & 1,545 & 2,229 & 2,229 \\ 
R$^{2}$ & 0.038 & 0.020 &  \\ 
Adjusted R$^{2}$ & 0.035 & 0.018 &  \\ 
$\rho$ &  &  & $-$0.998 \\ 
Inverse Mills Ratio &  &  & $-$0.695$^{***}$  (0.060) \\ 
Residual Std. Error & 0.676 (df = 1539) & 0.568 (df = 2223) &  \\ 
F Statistic & 12.106$^{***}$ (df = 5; 1539) & 9.207$^{***}$ (df = 5; 2223) &  \\ 
\hline 
\hline \\[-1.8ex] 
\textit{Note:}  & \multicolumn{3}{r}{$^{*}$p$<$0.1; $^{**}$p$<$0.05; $^{***}$p$<$0.01} \\ 
\end{tabular} 
\end{table} 

Model 3 (using Heckman selection) is the only one to correctly estimate $\beta_1$.  Model 1 is more accurate than Model 2, probably because mean imputation undermines the correlation between variables in the data.

\section{Question 9}
The counterfactual policy would result in a 1\% decrease in preference for union jobs, according to the model. This is probably not a very realistic model at all because we cannot account for all the factors involved in choosing a union vs. non-union job.

\end{document}
